\documentclass{beamer}
\usepackage{hyperref}
\usepackage{breakurl}

% Beamerによる数式フォントの置き換えを阻止
\usefonttheme{professionalfonts} 
% コマンドの説明は註18を参照
\usepackage[deluxe]{otf}
\usepackage[noalphabet]{pxchfon}
\setboldgothicfont{HaranoAjiGothic-Medium.otf} %\bfseriesの設定
\usepackage{bm}



\title{beamerでスライドを作ってみる}
\author{morimori12}  
\date{\today}

\begin{document}

\begin{frame}
  \titlepage
\end{frame}

\begin{frame}
  \frametitle{2枚目のスライド}
  \begin{itemize}
    \item 最初の項目
    \item 2番目の項目
    \item 最後の項目
  \end{itemize}
\end{frame}

\begin{frame}
  \frametitle{数式}
  真空中のマクスウェル方程式(微分形)は以下のように表される。
  \begin{align}
    \nabla \cdot \bm{E}  & = 0                                   \\
    \nabla \cdot \bm{H}  & = 0                                   \\
    \nabla \times \bm{E} & = -\dfrac{\partial\bm{B}}{\partial t} \\
    \nabla \times \bm{H} & = \dfrac{\partial\bm{D}}{\partial t}
  \end{align}
  ただし、$\bm{D}=\varepsilon_0 \bm{E},\bm{B}=\mu_0 \bm{H}$と表される。

  ※professionalfontsというフォントテーマを使うとBeamerによる数式フォントの置き換えを阻止することができる。\cite{takoyaki}
\end{frame}

% 参考文献
\begin{frame}
  \frametitle{参考文献}
  \bibliography{refs} %hoge.bibから拡張子を外した名前
  \bibliographystyle{junsrt} %参考文献出力スタイル
\end{frame}
\end{document}